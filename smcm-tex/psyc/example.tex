% arara: pdflatex
% arara: biber
% arara: pdflatex
% arara: pdflatex
% arara: clean: { files: [ example.aux, example.log ] }
% arara: clean: { files: [ example.out, example.toc ] }
% arara: clean: { files: [ example.bbl, example.bcf ] }
% arara: clean: { files: [ example.blg, example.ttt ] }
% arara: clean: { files: [ example.run.xml ] }
\documentclass{smcm-psyc-paper}

\bibliography{refs}

\title{Some Super Psychological Title}
\shorttitle{Psychology Title}
\abstract{This is an abstract!  Woohoo!}

\author{A. U. Thor}
\authornote{}
\affil{St. Mary's College of Maryland}


\date{August 27, 2013}


\begin{document}
\maketitle

\section{Captain America}
For as much progress that we've made,
  we've gone a little downhill
  these past fifty years or so \autocite{rogers:elements}.
I really don't know what much else to say,
  but I have this need to write some filler text
  so that something goes on to the next column.
Because columns are cool.

How cool?
\emph{Very} cool.
Let me \texttt{itemize} the ways!
Columns\dots
\begin{itemize}
\item make you look professional, and
\item fit more content onto one page without being mean to your eyes.
\item Did I mention they make you look professional?
\end{itemize}

For more reasons on why columns look professional
  (and what else looks professional),
  see \cite{compandtype}.
By the way, do you need some chemical equations?

\reaction{C12H22O11 + (H2SO+)4 -> 12C + 11H2O}

\section{Another!}
This document was written with \TeX! See also \cite{texbook}.
I don't really know anything about Captain America---sorry.

\begin{table}
  \centering
  \begin{tabular}{rll}
    \toprule
    Editor & Pros & Cons \\
    \midrule
    Emacs & Powerful & Steep \\
    Word  & Easy & Ugly, evil \\
    \bottomrule
  \end{tabular}
  \caption{An example table}
  \label{tab:ex}
\end{table}

\printbibliography
\end{document}
