\documentclass{psypaper}

\usepackage{expl3,xparse}

\ExplSyntaxOn

\NewDocumentCommand \ProgramName { m } {
  \textsl{#1}
}

\NewDocumentCommand \MicrosoftWord { } {
  Microsoft~\ProgramName{Word}
}

\NewDocumentCommand \MyEmail { } {
  \href{mailto:seallred@smcm.edu}{\texttt{seallred@smcm.edu}}
}

\NewDocumentCommand \wysiwyg { } {
  \textit{what-you-see-is-what-you-get}
}

\NewDocumentCommand \WYSIWYG { } {
  \textit{What-You-See-Is-What-You-Get}
}

\NewDocumentCommand \UseCanned { m } {
  \input{../../canned/#1}
}

\ExplSyntaxOff

%%% Local Variables: 
%%% mode: latex
%%% TeX-master: nil
%%% End: 


\title{An Conceptual and Utilitarian Introduction to \LaTeX}
\shorttitle{Introduction to \LaTeX}
\abstract{%
  \MicrosoftWord\ is a great tool for making a quick note or flyer,
    but it is not suited for academic or otherwise consistently-formed documents.
  While \MicrosoftWord\ does support `styles',
    it does not enforce (or even encourage) their use.
  In this short introduction, I introduce to you
    a new system for creating your papers---%
    a true tool to make your life easier (not to mention cheaper)---%
    called \LaTeX.%
}

\author{Sean Allred}
\authornote{If you have any questions or corrections,
  please open an issue on the GitHub repository
  or ask a nerd-friend to help you do so.}

\date{August 28, 2013}

\begin{document}
\maketitle

\section{What Is \LaTeX?}
First, some cold facts about \LaTeX:
\UseCanned{facts}
\UseCanned{qanda}

\LaTeX\ is not a \wysiwyg---it is not just another program you can use to write your papers.
While \LaTeX\ is often used by over-zealous nerds as a free alternative to \MicrosoftWord,
  it is more useful than that.
While most other programs you have seen for writing your documents
  are primarily concerned with how the finished product looks on the page (\wysiwyg),
  \LaTeX\ is \emph{only} concerned with the \emph{structure} of your document.
Who is the author?
What is its title?
Where are the sections?
\emph{These} are the questions you answer for \LaTeX;
  the rest it takes care of by itself.
You need not concern yourself with how it is presented on the page;
  the idea of \LaTeX\ is that \emph{it already knows how you want it laid out}.

Of course, \LaTeX\ is just a computer program.
\emph{Someone} has to tell it what to do.
This is where these files here come into play;
  the files in this collection tell \LaTeX\ what to do
  with your structured document.
At the time of writing, these files are very small extensions
  to the already widely-available collection of templates\footnote{We call them `document classes'.}
  available on the web.
As an example, if this paper looks like it has
  a familiar---while typographically dubious---format to you,
  there is a good reason: the majority of the templates in this collection
  are based upon a public template specifically designed
  to be compliant with the sixth edition of the APA Style Guide.
Since all APA-styled papers are supposed to have the same structure and layout,
  \LaTeX\ is \emph{the} tool for the job.
I need not know a \emph{single thing} about the APA style
  to write a document with perfect copmliance.

\section{Structured Documents}
By now you have heard me use the term `structured document' quite a bit.
What exactly do I mean by this?
In \MicrosoftWord, you generally lay out all of the elements of your document \emph{by hand}.
There is no convenient way to embed logical information---%
  things such as `this is the author' and the like---%
  into your paper while you're writing it.
Sure, recent versions of 
\end{document}

%%% Local Variables: 
%%% mode: latex
%%% TeX-master: t
%%% End: 
