% Every line that starts with % is a comment
\documentclass{article}
\usepackage{setup}

\begin{document}
\begin{figure}
  \centering
  \verbatiminput{fish-8.9mgL.dat} % Show our data file
  \caption{The data in \texttt{fish-8.9mgL.dat},
    an external file}
\end{figure}

\begin{figure}
  \centering
  \verbatiminput{fish-3.9mgL.dat} % Show our data file
  \caption{The data in \texttt{fish-3.9mgL.dat},
    an external file}
\end{figure}

\clearpage

\begin{table}
  \centering
  \pgfplotstabletypeset{fish-8.9mgL.dat} % Print out our data

  \caption{The table typeset directly from the data in
    \texttt{fish-8.9mgL.dat}}
\end{table}



\begin{table}
  \centering
  \pgfplotstabletypeset{fish-3.9mgL.dat} % Print out our data

  \caption{The table typeset directly from the data in
    \texttt{fish-3.9mgL.dat}}
\end{table}

\clearpage

\begin{figure}
  \centering
  \begin{tikzpicture}
    \begin{axis}[ybar, % This is a bar graph with the bars along
                       % the y axis
      title={Comparison of Opercullum openings
        with varying Oxygen content},        % title!!
      xlabel=Fish,                           % x axis label
      ylabel=Opercullum openings per minute, % y axis label
      legend pos=outer north east]
      
      \addplot+[error bars/y dir=both, % we want the y error bars
                                       % to go up AND down
                error bars/y explicit] % and we want to state their
                                       % column explicitly
        table[x=fishy, % our x axis is the fish
              y=mean,  % our y axis is the average openings per minute
        y error=stderror]  % and our y error is stderror
        {fish-8.9mgL.dat}; % and here is our data!
      \addlegendentry{\SI{8.9}{\milli\gram\per\liter}};      

      \addplot+[error bars/y dir=both,
                error bars/y explicit]
        table[x=fishy,
              y=mean,
        y error=stderror]
        {fish-3.9mgL.dat};
      \addlegendentry{\SI{3.9}{\milli\gram\per\liter}};
   \end{axis}
  \end{tikzpicture}
  
  \caption{A plot of the data in \texttt{measurements.dat}, with plot
    smoothing and error bars calculated directly from the original
    measurements.}
\end{figure}

\end{document}

%%% Local Variables: 
%%% mode: latex
%%% TeX-master: t
%%% End: 
